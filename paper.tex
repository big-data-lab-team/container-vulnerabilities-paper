% GigaScience template
\documentclass[a4paper,num-refs]{oup-contemporary}

\journal{gigascience}


%%%% Packages %%%%
\usepackage{siunitx}
\usepackage{minted} % Used for JSON highlighting
\usepackage{algpseudocode} % Algorithmic environment
\usepackage{xspace}
\usepackage{booktabs}
\usepackage{xcolor}
\usepackage{graphicx}
\usepackage{algorithm}
\usepackage{caption}
\usepackage{subcaption}
\usepackage{listings}
\usepackage{verbatim}
\usepackage{makecell}
\usepackage[flushleft]{threeparttable}

%%%% Commands %%%%
\newcommand{\todo}[1]{\color{red}\textbf{TODO:}#1\color{black}}
\newcommand{\note}[2]{\color{blue}Note: #1\color{black}}
\newcommand{\reprozip}[0]{ReproZip}


\title{Risks of Running Vulnerable Container Images for Data Analysis on HPC
Clusters}
  
\begin{document}

\author[1]{Bhupinder Kaur}
\author[1]{Aiman Hanna}
\author[1]{Tristan Glatard}

\affil[1]{Department of Computer Science and Software Engineering, Concordia University, Montreal, Canada}

\maketitle

\begin{keywords}
Containers; Security; Docker; Singularity; Neuroimaging.
\end{keywords}


\section{Introduction}

Virtualization, which was introduced in 1960s, refers to the creation of a virtual
version of something. Its meaning has evolved and took different forms since then, for
instance hardware virtualization, containerization, desktop virtualization, etc.
Here, in this paper our main focus is on containerarization and its security concerns,
specifically in neuroimaging field.
Containers, which provide operating system (OS) level virtualization, are very
popular. This is due to the need for fast development cycles, continuous delivery,
and cost reduction of infrastructure. Moreover, demand for an isolated
environment adds on to the popularity of containers. Indeed, containers accelerate
development cycles as compared to virtual machines and provide near-native performance.
They are very lightweight, flexible, and more resource-efficient than virtual
machines. By providing OS level virtualization they
removes the overhead of having an extra OS layer.

Although Linux Containers (LXC) and other alternatives such as Rocket were already
available, container-based virtualization gained a lot of interest when Docker was
introduced in 2013~\cite{gantikow2016providing}. Docker makes container's use more straightforward. However, there
are few limitations of Docker. For example, it is not supported by shared High Performance
Computing (HPC) environments because its default configuration runs containers
as root. Secondly, Docker containers cannot be used for reproducibility as Docker
images can be updated, and hence losing specific package versions required for
reproducing scientific results. Consequently, Singularity was introduced in 2016 with the main
goal of removing drawbacks of Docker. Singularity was developed mainly to
support HPC environments and to provide reproducibility.

No doubt containers have gained a lot of popularity after Docker, however containers tightly
integrate with the host due to the sharing of the kernel, which raises security
concerns. Figure \ref{fig:container-overview} provides overview of the containers architecture.
They also share three main security threats with other virtualization
techniques~\cite{gantikow2016providing}.

\begin{figure}
  \centering
  \includegraphics[width=.7\columnwidth]{Figures/container.png}
  \caption{Architecture of Container-based
                Virtualization.}
  \label{fig:container-overview}
\end{figure}

\textbf{Privilege Escalation} A malicious attacker can break out of the container
and gain control of the host system and other containers running on the same host.

\textbf{Denial-of-Service} One container may end up eating all the resources of the
host hence resulting in the starvation of the host and other containers.

\textbf{Information Leak} Private data of the host and other containers could be
leaked and can be used for further attacks.

However, two features of the Linux kernel, \textit{control groups} and \textit{namespaces},
are used by containers to mitigate these threats to a certain extent. Security
of containers is still most concerning issue.
Except web applications, containers are also popularly used for scientific computing
and data science in neuroscience. In fact, Singularity was developed for scientific applications
by keeping focus on HPC environments.
The goal of this paper is to report all vulnerabilities that are present in
container images, specifically used by neuroimaging field, and access their
related risks for data analysis.

The existing work on containers focuses mainly on the security of Docker
containers.
This focus is justified by the fact that, containers expose the host's resources
(e.g., file system/ IPC) to the guest system. This feature raises a confidentiality
threat for the applications running on the same host. Previous studies evaluated
security of Docker engine ~\cite{martin2018docker, sultan2019container, combe2016docker, bui2015analysis},
and have scanned vulnerabilities on Docker hub~\cite{Shu2017, gummaraju2015over}.
Also Singularity came up with Stools, which can be used for scanning Singularity images
for security and quality checks.
Here, we focus on the specific context of scientific data analysis on HPC clusters, taking
neuroimaging as an example.

This paper is organized as follows: in next section, we provide description
of materials and methods which includes summarizing a particular use case of
containers in neuroscience, which is used for scientific computing. It involves
usage of containers on shared HPC cluster. Additionally, in this section we
describe all the resources that are used by this use case. We also explain
scanning tools that we use for scanning container images to report
vulnerabilities. Next section of this paper is about the results. Here, we
present number and type of vulnerabilites that are present in the container images
which are used for data analysis on shared HPC clusters, and discuss how many of them are
critical vulnerabilities. Following section is about discussion of these results
where we provide more insight into the risks that are caused by these vulnerabilties
and how they can be leveraged to design attacks. Here, we also provide some such
vulnerability examples. Finally, we conclude this paper by discussing ten simple
rules that can be followed to secure container image for use in clusters.

\section{Materials and Methods}

In this section we summarize a particular use case of containers, Canadian
Open Neuroscience Platform (CONP). It is a national platform for sharing neuroscience
research data and aims at creating an interactive interface for neuroscientists and
clinical neuroscience. The main goal of the CONP platform is to store, process, and
distribute massive amounts of data produced by modern neuroscience.
Here, we summarize all the resources that are
used by CONP to function.

\subsection{Compute Canada Clusters}

Compute Canada is a national High Performance Computing (HPC) research platform.
It provides infrastructure and services for Advanced Research Computing (ARC).
Its goal is to bring excellence, ease, and innovation in research of Canadian
research institutions by providing storage, computing power, and required
softwares to carry out research.
It integrates with four regional partners namely ACENET, Calcul Québec, Compute
Ontario, and WestGrid for providing local support.
Compute Canada uses Slurm Workload Manager to schedule jobs and Lustre shared
file system to provide high bandwidth to its users.
Control groups are used by it to limit number of resources that can be used
by a particular user.
Compute Canada do not support Docker as its default configuration
runs containers as root. However, Compute Canada supports Singularity
containers because it removes the drawback of Docker.

\subsection{Container Engine: Singularity}

Singularity, which is introduced in 2016, provides operating system (OS) level
virtualization.
It offers mobility of compute by facilitating portable environments 
through a single image file~\cite{kurtzer2016singularity}. Once Singularity image
is created, it is hashed by using SHA256 hashing. Hence, it cannot be changed
once it is created, consequently, it can be used for reproducing the results of
scientific experiments. Singularity was introduced with the main goal of
providing support for multi-user environment. Main features provided by
Singularity are mobility of compute and reproducibility~\cite{kurtzer2017singularity}.
Mobility of compute is defined as creating and maintaining a workflow on
a local machine that can be easily ported on other hosts, Linux distributions,
and/or cloud service providers. To achieve this, Singularity packs everything
that is required for the application to run in a distributable image. Then that
image can be copied, archived, and shared. Moreover, Singularity containers
are easily portable across different versions of the C library and different kernel
implementations.
Same features of Singularity, which are used for mobility of compute,
facilitate reproducibility as well~\cite{sochat2017enhancing}.

\subsection{Container Image Formats}

A container image is a package that contains the application alongwith its
required dependencies and libraries.
Nowadays, two most popular containerization tools used are Docker and Singularity.
Both have different image format. Below we provide description of both image
formats.

\subsubsection{Docker Image Format}

Docker images are organized as a series of layers which are stacked on top
of each other. Each layer consists of a filesystem diff that is introduced
due to the changes made on the layer below it. All layers of the Docker image
are identified by unique layerIDs. All these layers stacked together gives a
unified and complete view of a Docker image. These layers are
compressed into a single image and can be pushed to the Docker Hub.
There are two ways to create Docker image. The first is to use an existing
Docker image as a base image and then build a new image from that one, by
making changes in the filesystem, which is saved as a new layer of the
Docker image. The second way is to build a new Docker image from scratch.

\subsubsection{Singularity Image Format}

Singularity takes snapshot, locks, and archives the developed application, which
is known as Singularity image. Due to image hashing, Singularity does not
contain image layers, unlike Docker images. So Singularity image is a single
immutable file that can be moved around as any other file. While publishing experimental
results, authors can also publish Singularity image along with its hash, which
allow other researchers to verify results.

\subsection{Application frameworks}

Following are the application frameworks in neuroscience that are using containers
on HPC clusters.

\subsubsection{Boutiques}

Boutiques is a application sharing system~\cite{glatard2015boutiques}. It is based on Linux containers
and can be used to publish, integrate, and execute third-party
applications across different platforms automatically. Boutiques framework is
employed by CONP to integrate and execute applications on CBRAIN web portal.
With Boutiques, the command-line of the application is
described through a flexible template, which is called descriptor. Descriptor
describes the input that is required by the application and the output it produces~\cite{glatard2018boutiques}.
It points to a Docker or Singularity image, where the application is installed.
Developers of the application produce Boutiques descriptor, and stores it along
with the application. Boutiques descriptor are described in JSON format. There
is a set of core tools which help in constructing, validating and executing bou-
tiques descriptors. During the application run-time, the execution platform
builds a command line by using input values given by the user and boutiques
descriptor of the application.
Currently, Boutiques tools use 23 different container images. Out of
which 7 are Singularity images and rest are Docker images.

\subsubsection{Brain Imaging Data Structure (BIDS) Apps}

BIDS apps~\cite{gorgolewski2017bids} is a framework designed to share and execute neuroimaging
analysis pipelines. It improves usability, accesibility, and reproducibility
of different operating system users as well as HPC users through containerization.
BIDS apps share reproducibility goal with Boutiques, though they are conceptually different.
BIDS apps main goal is to standardize application interfaces whereas Boutiques goal is to
make application sharing and integration easier with different platforms.
Currently, there are 27 different container images that are used by
BIDS App. All these are Docker images.

\subsubsection{ReproNim}

ReproNim is a Center for Reproducible Neuroimaging Computation. It is a vision
to help neuroscientists in reproducing neuroimaging research. Repronim
uses RepronIn, BrainVerse, Neuroimaging Computation Environments Manager (NICEMAN),
and NeuroBlast software tools for its
functioning. ReproIn is a software that automate acquisition and conversion
of collected MRI data to BIDS standard format with DataLad version management~\cite{kennedy2019everything}.
BrainVerse is a cross-platform software framework and collaborative desktop application
that helps in managing, tracking, and sharing information.
NICEMAN is a software system that tracks and manages available computation resources
and makes their use in a scalable and reproducible way.
Neuroblast is a service that facilitates data sharing of existing studies, and it also
helps users to search similar studies based on combination of analysis,
tasks and activation patterns. There are currently 28 different Singularity
images used by Repronim. 

\subsection{Image Update Process}

If there exist vulnerabilities in the container image, then there are two ways
to proceed, either not to run that image at all, or run only after updating the
image. Updating the image involves updating all software packages that are
inside that image. However, this option effects the reproducibility
of the image, which means that the image cannot be used to verify research
findings of other scientists. An alternative approach can be to trim unnecessary
packages from images, which in turn reduces the number of vulnerabilities
present in images.


\subsection{Scanning tools used}

There are various scanning tools that are available for scanning container
images like Vuls, Clair, ClairScanner, Anchore, Dagda, etc.
Out of these we take Anchore, Vuls, and ClairScanner tools for our experiments to scan container
images which are used by the above discussed application frameworks.
For scanning Singularity images, Stools scanner is used, which is specifically designed
for Singularity images. Stools internally make use of Clair-Scanner for
scanning.

\subsubsection{Anchore}

Anchore is an end-to-end container security, and open-source compliance plat-
form. Anchore is a static scanning tool that analyses container images and
lists operating system packages,
unofficial packages, configuration files, and language modules [58]. In other
words, Anchore is a set of tools, which users can employ for scanning container
images and applying custom policies to them. It provides visibility and trans-
parency of the container environment. Anchore offers services in two ways:
Anchore Engine and Anchore Engine Command Line Interface (CLI).
Anchore Engine is available as a Docker image, which is available on Docker
Hub. Anchore Engine CLI provides command line interface in addition to An-
chore Engine REST API. Anchore Engine CLI can be installed directly on the
host and used to inspect images.

\subsubsection{Vuls}

Vuls is an open-source vulnerability scanner for Linux/FreeBSD, written in Go
language. It is an agentless vulnerability scanner and uses various
vulnerability databases. Using dynamic scanning technique, it offers
remote and local scan. Remote scan involves setting up only one machine, to
which other target servers are connected via SSH. Local scan involves running
the scan on the target server itself. Further, Vuls offer fast scan, fast root scan,
and deep scan.
Fast scan is performed without root privileges, whereas fast root scan is
performed with root privileges. The latter also uses checkrestart utility to de-
tect processes that are updated but not yet restarted because if they are not
restarted, updates are not effective. So fast root scan also detects these pro-
cesses. Both fast scan and fast root scan are performed offline without Internet
access, and put almost no load on the scan target server. Deep scan uses
root privileges for scanning, and is used to get a more detailed scan. It checks
changelogs, which puts more load on the target server. As shown in Figure 4.3
it is clear that vulnerability detection time of Vuls is much better than An-
chore. Vuls tool uses dynamic scanning, so the scanning time does not depend
on the image size.
Graph 4.5 shows percentage of vulnerabilities that are
covered by Vuls and Anchore tools. This graph can be explained with the help
of List 4.4, which shows vulnerability databases that are referred, by both Vuls
and Anchore. Anchore covers less percentage of vulnerabilities as compared
to Vuls because Anchore relies mainly on vulnerability databases published by
Linux distributions, whereas Vuls additionally refers to another sources such as
databases from Japan, US-CERT, etc.

\subsubsection{Stools}

Stools combines CoreOS's ClairScanner with the Continuous Integration (Travis and
Circle) for scanning Singularity images for vulnerabilities.
Initially, ClairScanner was intented to scan Docker Containers. This tool does not fit well
to scan Singularity images because of the different image formats. Docker layers
consist of layers, whereas Singluarity images consist of a single binary file.
However, TravisCI and CircleCI are commonly used for testing code, and these
services also offer running containers. So Stools combined both the concepts.
It involves spinning a ClairScanner container during testing and building a
Singularity image, and scanning it for filesystem before the image is finalized.


\subsection{CBRAIN}

\href{http://github.com/aces/cbrain}{CBRAIN}~\cite{sherif2014cbrain} is a web portal that is
used for the processing of distributed data on various storage locations of computing
clusters. CBRAIN system deployed by Montreal Neurological Institute depends on the infrastructure
delivered by Compute Canada~\cite{das2016mni}.
It is also used by CONP to exploit services of Compute Canada. CBRAIN is a collaborative
neuroimaging research platform that is active in production since 2009. CBRAIN
provides transparent access to various distributed storage sites and computing
resources across the world. CBRAIN’s infrastructure consists of three main
layers, namely access layer, service layer, and infrastructure layer~\cite{sherif2014cbrain}. The access
layer is for CBRAIN users, and can be accessed through any modern browser or
RESTful API. The service layer consists of CBRAIN portal, which contains
metadata and databases. It provides information about users, their permissions,
resources etc. The infrastructure layer consists of data resources and computing
resources, which are connected through a network. 
CBRAIN supports Docker, Singularity, BIDS apps, and Boutiques.
CBRAIN users uses a portal
account on the cluster. Users are only able to
run pre-defined applications on the cluster through the CBRAIN portal.

%\bibliographystyle{vancouver-authoryear}
\bibliography{bibliography}


\end{document}

