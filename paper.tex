% GigaScience template
\documentclass[a4paper,num-refs]{oup-contemporary}

\journal{gigascience}
%%%% Packages %%%%
\usepackage{siunitx}
\usepackage{algpseudocode} % Algorithmic environment
\usepackage{xspace}
\usepackage{csvsimple}
\usepackage{minted} % Used for JSON highlighting
\usepackage{datatool}
\usepackage{booktabs}
\usepackage{xcolor}
\usepackage{amsmath}
\usepackage{graphicx}
\usepackage{multirow}
\usepackage{array}
\usepackage{algorithm}
\usepackage{caption}
\usepackage[justification=centering]{caption}
\usepackage{subcaption}
\usepackage{listings}
\usepackage{verbatim}
\usepackage{adjustbox}
\usepackage{makecell}
\usepackage{titlesec}
\usepackage{anyfontsize}
\usepackage[flushleft]{threeparttable}
\usepackage{xspace}

%%%% Commands %%%%
\newcommand{\todo}[1]{\color{red}\textbf{TODO:}#1\color{black}}
\newcommand{\change}[2]{\color{cyan}Changes: #1\color{black}}
\newcommand{\reprozip}[0]{ReproZip\xspace}
\newcommand{\rom}[1]{\lowercase\expandafter{\romannumeral #1\relax}}
\newcommand{\tristan}[1]{\color{red}From Tristan: #1\color{black}}

\renewcommand{\labelitemi}{$\textendash$}
\title{An Analysis of Security Vulnerabilities in Container Images for Scientific Data Analysis}
  
\begin{document}

\author{Bhupinder Kaur}
\author{Mathieu Dugr\'e}
\author{Aiman Hanna}
\author{Tristan Glatard}

\affil{Department of Computer Science and Software Engineering, Concordia University, Montreal, Canada}

\maketitle

\begin{abstract}
Software containers greatly facilitate the deployment and reproducibility
of scientific data analyses on high-performance computing clusters
(HPC). However, container images often contain outdated or unnecessary
software packages, which increases the number of security vulnerabilities
in the images and widens the attack surface of the infrastructure. This
paper presents a vulnerability analysis of container images for scientific
data analysis. We compare results obtained with four vulnerability
scanners, focusing on the use case of neuroscience data analysis, and
quantifying the effect of image update and minification on the number of
vulnerabilities. We find that container images used for neuroscience data analysis
contain hundreds of vulnerabilities, that software updates remove about two
thirds of these vulnerabilities, and that removing unused packages is also
effective. We conclude with recommendations on how to build container
images with a reduced amount of vulnerabilities.

\end{abstract}

\begin{keywords}
Containers; Docker; Singularity; Security Vulnerabilities; Neuroimaging.
\end{keywords}

\section{Introduction}

Software containers have emerged has an efficient solution to deploy
scientific data analyses on HPC clusters, due
to their portability, ease of use, and limited overhead. On HPC systems, the
Singularity~\cite{kurtzer2017singularity} framework is often preferred to
\href{http://docker.com}{Docker} due to its
secure handling of multi-user environments and convenient support for
Docker images. Singularity is now available in dozens of HPC
clusters around the world and routinely used for Big Data analysis.

Taking advantage of core Linux kernel features such as namespaces, control
groups and chroot, containers isolate processes from the host computer, and
control the memory, CPU, network and file-system resources assigned to
them. However, containers still share the kernel, mounted file systems and
some devices with the host, which raises security
concerns~\cite{martin2018docker, sultan2019container, combe2016docker} and
opens the door to privilege escalation, denial of service, information leak
and other types of attacks~\cite{gantikow2016providing}. 

Container images typically include full operating system (OS) distributions in
addition to data analysis software and their dependencies. They are rarely
updated due to concerns that software updates will interfere with
the results~\cite{gronenschild2012effects, glatard2015reproducibility}.
Images also typically include more dependencies than required, to make them
easier to reuse between experiments. As a result, over 30\% of official
images in DockerHub have been shown to contain high-priority security
vulnerabilities~\cite{gummaraju2015over}, images on average contain over
180 vulnerabilities~\cite{Shu2017}, and vulnerabilities are often caused by
outdated packages~\cite{zerouali2019relation}.

In this study, we focus on the vulnerabilities present in container images
deployed on HPC clusters for scientific data analysis, in particular in the
neuroimaging domain. We address the following questions:

\textit{What is the current amount of vulnerabilities in
container images deployed on HPC clusters?} Vulnerabilities are possible
attack vectors that can seriously compromise the security of HPC clusters
and the integrity of user data. We report vulnerability scans produced
by four popular image scanning tools: Anchore, Vuls, Clair, and Singularity tools.

\textit{Can the amount of vulnerabilities be reduced by updating the images?}  
For many reasons related to reproducibility and the lifecycle of research
projects, container images deployed on HPC clusters often include outdated
software. We report on the effect of software updates on the amount of
vulnerabilities found in images.

\textit{Can the amount of vulnerabilities be reduced by minifying images?} 
Container images often include more software packages than necessary for 
a typical analysis. We report on the impact of unused software packages on
the presence of vulnerabilities.

The remainder of this paper 
describes the container images and scanners used in our experiment, and our methodology for updating and minifying images.
Results present the vulnerabilities detected in container
images, quantify the effectiveness of updating and minifying images, and
explain the differences observed between scanners. In conclusion, we
provide a set of image creation guidelines for a more secure deployment of
containers on HPC clusters.

\section{Materials and Tools}

We used container
images from two popular application frameworks, as well as
four of the major image scanners.

\subsection{Container Images}

We scanned all container images available at the time of this study on two containerization frameworks
used in neuroscience: BIDS
apps~\cite{gorgolewski2017bids} (26 images) and Boutiques~\cite{glatard2018boutiques} (18 images),
totalling
44 container images. At the time of the study, BIDS apps had 27 images,
out of which one wasn't available on DockerHub. Boutiques had 49 images,
%listed using Boutiques \texttt{bosh search} command,
however, only 23 unique images were listed, out of which 3 couldn't be retrieved and 2
were already included in BIDS apps. All the final 26 images
from BIDS apps were Docker images, whereas the 18 Boutiques images contained 12 Docker images
and 6 Singularity images.

\subsection{Image Scanners}

We compared the results obtained with four container image scanners: Anchore, Vuls, and
Clair to scan Docker images, and Singularity Container Tools
(Stools) to scan Singularity images. 

\href{https://github.com/anchore/anchore-engine}{Anchore} is an end-to-end, open-source container security platform. It
analyzes container images and lists vulnerable OS
packages, non-OS packages (Python, Java, Gem, and npm), and files.
In our experiments, we used Anchore Engine version 0.5.0 through Docker image \texttt{anchore/anchore-engine:v0.5.0}, and
Anchore vulnerability database version 0.0.11.

\begin{table*}
\begin{tabular}{|c|c|c|c|}
 \hline
\textbf{OS} &	\textbf{Anchore} &	\textbf{Vuls} &	\textbf{Clair} \\
\hline
	\textbf{Alpine} & Alpine-SecDB &	Alpine-SecDB &	Alpine-SecDB \\
\hline
	\textbf{CentOS} & Red Hat OVAL Database & Red Hat OVAL Database and Red Hat Security Advisories & Red Hat Security Data \\
\hline
	\textbf{Debian} & Debian Security Bug Tracker &	Debian OVAL Database and Debian Security Bug Tracker & Debian Security Bug Tracker \\
\hline
	\textbf{Ubuntu} & Ubuntu CVE Tracker &	Ubuntu OVAL Database &	Ubuntu CVE Tracker \\
 \hline
\end{tabular}
\caption{Vulnerability databases used by scanners for different OS distributions. All scanners also refer to 
the National Vulnerability Database (NVD) for vulnerability metadata.}
\label{table:databases}
\end{table*}

\href{https://github.com/future-architect/vuls}{Vuls} is an open-source vulnerability scanner for Linux and FreeBSD. It
offers both static and dynamic scanning, and both local and remote
scanning. In our experiments, we used Vuls 0.9.0, executed through Docker image
\texttt{vuls/vuls:0.9.0} in remote dynamic mode.

\href{https://github.com/quay/clair}{Clair} is an open-source and
extensible vulnerability scanner for Docker and appc container images,
developed by CoreOS (now Container Linux), a Linux distribution to deploy
container clusters.
%  Clair has a client-server architecture, in which the
% server scans Docker images layer by layer and maintains a database of
% vulnerabilities.
 We used Clair through
\href{https://github.com/arminc/clair-scanner}{Clair-scanner}, a tool to
facilitate the testing of container images against a local Clair server. 
% Clair-scanner scans the image,
% prepares a list of vulnerabilities, compares that list against a
% whitelist, and flags vulnerabilities that are not present in the whitelist.
% In our experiments, we did not use a whitelist to filter scanning results in
% order to make a fair comparison between scanners.
% Clair-scanner maintains a Docker image with the up-to-date vulnerability
% database from a set of different sources.
% Figure~\ref{database} lists
% vulnerability databases that Clair referred for Alpine, Ubuntu, Debian, and Centos.
% Clair also uses NVD database for getting vulnerability metadata.
We used Clair version 2.0.6, executed through
Docker image \texttt{arminc/clair-local-scan:v2.0.6}. For the vulnerability
database, we used Docker image \texttt{arminc/clair-db:latest}, last
updated on 2019-09-18.

\href{https://github.com/singularityhub/stools}{Singularity Tools} (Stools)
are an extension of Clair for Singularity images. Stools
exports Singularity images to \texttt{tar.gz} format, acting as a single layer Docker image
to circumvent the Docker-specific requirements in the Clair API.
In our experiments, we used Singularity Tools version 3.2.1 through Docker
image
\texttt{vanessa/stools-clair:v3.2.1}.
% , which is using another Docker
% image \texttt{arminc/clair-db:latest} for referring to vulnerability feed.
Since Stools uses Clair internally for scanning, the vulnerability databases used
by Stools are the same as mentioned for Clair.
To scan Singularity images, we followed the steps mentioned in the
\href{https://github.com/singularityhub/stools}{Stools documentation}.

\subsection{Vulnerability Databases}

Scanners refer to two types of
vulnerability databases (Table~\ref{table:databases}). The first one is the Open Vulnerability and
Assessment Language (OVAL) database, an international open standard that
supports various OS distributions including Ubuntu, Debian and CentOS but
not Alpine. The second one are vulnerability databases from specific OS
distributions, such as Alpine-SecDB, Debian Security Bug Tracker, Ubuntu
CVE Tracker, or Red Hat Security Data. In these databases, OS distributions often assign a
status to each vulnerability, to keep track of required and available
security fixes in different versions of the distribution. Vuls uses OVAL
databases for all distributions except Alpine. On the contrary, Clair exclusively refers to
distribution-specific databases. Anchore uses OVAL only for CentOS, as distribution-specific databases
are assumed to be more complete.
It is also worth noting that there is no vulnerability data
for Ubuntu 17.04 and 17.10 distributions in the OVAL database, since these
distributions have reached end of life, meaning that images with these
distributions cannot be scanned with Vuls.
%  \tristan{why doesn't vuls use
% Ubuntu CVE tracker?}.\change{They have it
% in their TODO list}.
% \tristan{In Vuls, what happens when 2 databases are used?}.
% \change{Using OVAL database alone is not sufficient because
% OVAL does not contain all vulnerabilities as the main intention
% of the OVAL database is to provide a detailed description of the most recent vulnerabilities}.

For CentOS images, Anchore and Clair give scanning results using Red Hat
Security Advisory (RHSA) identifiers, whereas Vuls uses the Common
Vulnerabilities and Exposures (CVE) identifiers used in OVAL. We mapped
RHSA identifiers to corresponding CVE identifiers, to allow for a
comparison between scanners.

Different vulnerabilities may be reported by scanners if scanning
experiments take place on different dates. To avoid such discrepancies, we
froze the vulnerability databases used by these scanners as of 2019-09-25.
% \tristan{this contradicts the date mentioned in the paragraph on Clair}.
% \change{No, its not. Because the date mentioned in Clair is the date
% when Clair's database was last updated (because its database is present as
% a Docker container so we fetched that was latest updated). However, here the date represents
% the date when we
% froze the vulnerability databases for all the scanners}.

\subsection{Image Update}

A first approach to reduce the number of vulnerabilities in container
images is to update their packages to the latest version available in the OS
distribution. To study the effect of such updates, we developed a script
(available
\href{https://github.com/big-data-lab-team/container-vulnerabilities-paper/blob/master/Scripts/update}{here})
to identify the package manager in the image, and invoke it to update all
OS packages. We updated images on 2019-11-05.

\subsection{Image Minification}

A second approach to reduce the number of vulnerabilities in the images is
to remove unnecessary packages, an operation potentially specific to each
analysis. We used the open-source \reprozip tool~\cite{rampin2016reprozip}
to capture the list of packages used by an analysis. \reprozip first
captures the list of files involved in the analysis, through system call
interception, then retrieves the list of associated software packages, by
querying the package manager. We extend this list with a passlist of
packages required for the system to function, such as \texttt{coreutils}
and \texttt{bash}, and with all the dependencies of the required packages,
retrieved using
\href{http://manpages.ubuntu.com/manpages/xenial/man1/debtree.1.html}{Debtree}.
\href{https://linux.die.net/man/1/repoquery}{Repoquery} could be used in
RPM-based distributions instead. Our minification script, available
\href{https://github.com/big-data-lab-team/container-vulnerabilities-paper/tree/master/Scripts/minification}{here},
installs \reprozip in the image to minify, runs an analysis to collect a
\reprozip trace, and finally deletes all unnecessary packages. We had used
the \href{https://github.com/ReproNim/neurodocker}{Neurodocker} tool
initially, but it did not affect the detected vulnerabilities
as it was removing unused files without using the package manager.

Using this approach, we minified five Debian- or Ubuntu-based BIDS app images,
using basic analysis examples found in the applications documentation.

\section{Results}

Figure~\ref{fig:vulnerabilities} presents our results. All the collected
data are available in our GitHub repository at
\url{https://github.com/big-data-lab-team/container-vulnerabilities-paper}
with a Jupyter notebook to regenerate the figures. 

\subsection{Detected Vulnerabilities}


\begin{figure*}
\includegraphics[width=\textwidth]{Figures/results.pdf}
\caption{\label{fig:vulnerabilities} Number of vulnerabilities detected by
Anchore and Stools in container images. \textbf{(A)} Number of vulnerabilities by
container image and severity, showing hundreds of detected vulnerabilities
per image. Images s*,t*,u*,v*,w* and x* are Singularity images scanned by Stools. \textbf{(B)} Effect of image minification and
package update on 5 container images, showing that both techniques are
complementary \textbf{(C)} Number of vulnerabilities by number of
packages, showing a strong linear relationship \tristan{add Singularity images and update regression parameters accordingly.}. \textbf{(D)} Number of
vulnerabilities by number of packages \emph{after package update}, showing that software updates 
importantly reduce the number of detected vulnerabilities.}
\end{figure*}

An important amount of vulnerabilities were found in the tested container
 images (Fig~\ref{fig:vulnerabilities}-\textbf{A}), with an average of 468 vulnerabilities
  per image and a median of 372 \tristan{update values with singularity images}. In comparison, no vulnerabilities were found in base
 Docker images \texttt{ubuntu:20.04} and \texttt{centos:7} after package
 update. Moreover, a significant fraction of detected vulnerabilities are
 of high severity
 (\href{https://www.first.org/cvss/specification-document}{CVSS} score
 >=7.0) and a few of them are of critical severity (CVSS >= 9.0). Remote
 attackers could possibly exploit these vulnerabilities to execute
 arbitrary code in the container, by crafting responses to specific network
 requests. Images based on the Alpine distribution 
 had the lowest numbers of vulnerabilities, but no significant difference
 in the numbers of vulnerabilities detected in 
 Ubuntu, Debian or CentOS distributions was observed.

Unsurprisingly, a strong linear relationship is found between the number of
detected vulnerabilities and the number of packages present in the
image (Fig~\ref{fig:vulnerabilities}-\textbf{C}, r=0.83,
$p\textless10^{-10}$) \tristan{update values with singularity images}. On average, 1.8 vulnerabilities are introduced for
each new package installation. This observation motivates a systematic
review of software dependencies by application developers, to avoid
unnecessary packages in container images. This is also an argument in favor of lightweight
distributions such as Alpine. Compared to Ubuntu and Debian distributions,
CentOS images seem to have a lower number of vulnerabilities by package on
average, although data is too scarce to conclude.

\subsection{Effect of image update}

Updating container images reduces the number of vulnerabilities by package
by a factor of 3 on average, resulting in only 0.6 extra vulnerabilities by
package (Fig~\ref{fig:vulnerabilities}-\textbf{D}, r=0.81,
$p\textless10^{-7}$). Twelve container images are missing on this figure:
six of them could not be updated due to various issues with the package
manager, and six of them are Singularity images that we didn't update.
% \change{3 out of these missing 6 images retured an error while updating which
% indicates a problem with the package installer, 2 were unable to update because they
% reached EOL and last one with CentOS 7.1.1503 returned libselinux conflicts with systemd} 
In spite of the associated reproducibility challenges, updating
packages therefore appears to be an efficient way to avoid vulnerabilities. It
is not an ultimate solution though, as a substantial number of
vulnerabilities remain.

\subsection{Effect of minification}

Another approach to reduce the number of vulnerabilities involves deleting
unnecessary packages from the container images. It is a tedious operation,
as it requires running an actual data analysis in the container image, to
identify the packages required by the application. In addition, the
resulting container image is only valid for the specific type of analysis
used in the minification process, as other executions might require a
different set of packages. 

Using the ReproZip-based approach described previously, we minified 5
different images covering the spectrum of detected vulnerabilities
(Fig~\ref{fig:vulnerabilities}-\textbf{B}). We find that minification reduces the
number of vulnerabilities, albeit less systematically than package update.
For some container images, such as image \textbf{S}, minification removes more
than 70\% of the detected vulnerabilities. For other images, such as
image \textbf{g}, it only reduces the number of vulnerabilities by less than 1\%.
The effect of minification stems from the number of packages
that can be removed, which varies greatly across images. For
instance, images \textbf{g} and \textbf{a} have a large number of packages,
but the last majority of them is required by the analysis, which makes
minification less useful. In other cases, a limited number of unnecessary packages contain 
a significant number of vulnerabilities, which makes minification very impactful. 
This was the case in images \textbf{d}, \textbf{S} and \textbf{U}, where removing compilers
and kernel headers reduced the number of vulnerabilities by an important fraction. 

\subsection{Combined effect of image update and  minification}

Package update and image minification remove different types of
vulnerabilities. The former is efficient against vulnerabilities that have
been fixed by package maintainers, while the latter targets unused
software. In two of the five tested images (images \textbf{S} and \textbf{U}), we find that combining update
and minification further reduces the number of vulnerabilities compared to
using only one of these processes
(Fig~\ref{fig:vulnerabilities}-\textbf{B}). \

\subsection{Differences between scanners}

The results presented so far were obtained with Anchore (Docker images) and
Stools (Singularity images). We scanned the Docker images with two other tools,
Clair and Vuls, to evaluate the stability of our results. Important
discrepancies were found between scanners (Fig~\ref{fig:venn}), in
particular between Anchore and the other two scanners, for which Jaccard
coefficients as low as 0.6 were found, meaning that scanning results only
overlapped by 60\%. Vuls and Clair appear to be in better agreement, with a
Jaccard coefficient of 0.8.

% where do the discrepancies come from?
We analyzed these results and explained some reasons behind the observed
discrepancies. Out of 4453 vulnerabilities detected by Anchore only (region
\textbf{1} in Fig~\ref{fig:venn}), 4443 are found in the development
package of the C library (\texttt{linux-libc-dev} in Ubuntu and Debian).
Clair detects only Debian vulnerabilities in \texttt{linux-libc-dev},
whereas Vuls do not detect vulnerabilities in this package at all. Since Anchore
ignores Debian
vulnerabilities flagged as \texttt{minor}, it
might either detect (region \textbf{2}) or ignore (region \textbf{3})
the
Debian vulnerabilities detected by Clair in \texttt{linux-libc-dev}. The
remaining 10 vulnerabilities in region \textbf{1} are found in sub-packages
of vulnerable packages: they are correctly reported by Anchore and missed
by Vuls and Clair. 

Many vulnerabilities in region \textbf{3} and \textbf{4} are from images
based on Ubuntu 14.04. In the Ubuntu CVE tracker database used by Clair and
Anchore, there are two entries for Ubuntu 14.04: one for LTS (Long-Term
Support), a Ubuntu release with 5 years of technical support, and another one
for ESM (Extended Security Maintenance), a release that provides security
patches beyond the 5 years covered by LTS. Although all the scanned images
are LTS, Clair refers to the ESM database entry while Anchore and Vuls refer to the
LTS database entry. The vulnerabilities present in region \textbf{3} due to
this discrepancy are incorrectly missed by Anchore and Vuls: they have been
detected in ESM but were already present in LTS. The vulnerabilities in
region \textbf{4} are incorrectly missed by Clair: they have been fixed in
ESM but are still present in LTS. 

Some vulnerabilities in region \textbf{6} are due to bugs in Anchore: the
\textit{epoch bug} ignores vulnerabilities related to package versions that
contain an epoch (\texttt{:}); the \textit{out of standard bug} ignores
vulnerabilities that are ignored by the Ubuntu distribution. We reported
these bugs to the Anchore developers through their Slack channel. Some
vulnerabilities in region \textbf{6} are also due to the fact that Anchore
intentionally ignores Debian vulnerabilities flagged as \texttt{minor}.

Finally, 32 vulnerabilities that are flagged temporary by the Debian
distribution are reported by Vuls but not by Anchore or Clair (region
\textbf{7}). The remaining 504 vulnerabilities in this region are all found
in CentOS images. We weren't able to explain why they were detected by Vuls
only.

% \begin{table*}
% 	\csvreader[%
% 	 tabular={|c|c|c|c|c|c|},
% 		table head = \hline\textbf{Labels} &\textbf{{Image names}} & \textbf{Anchore} & \textbf{Clair} & \textbf{Vuls} & \textbf{Stools}\\\hline,
% 		late after line= \\,
% 			late after last line=\\\hline %
% 		]{resul.csv}{labels=\labels,names=\names,A=\A,C=\C,V=\V,Stools=\Stools}%
% 			{\labels & \names & \A & \C & \V & \Stools}
% 		   \centering
% 		\vspace*{1mm}
% 		\caption{\label{table1}Number of vulnerabilities detected by the four scanners. Vuls shows \textit{no data} for images \texttt{bids/antscorticalthickness:v2.2.0-1} and \texttt{bids/afni\_proc:latest}
% 		since their OS distributions reached end of life.} 
% 	\end{table*}
	\begin{figure}
        \includegraphics[width=\columnwidth]{Figures/venn.pdf}
	\caption{\label{fig:venn} Differences between vulnerabilities detected
	by the different scanners. The Jaccard coefficients between the sets of
	detected vulnerabilities are quite low, showing important discrepancies
	between the scanners: Jaccard(Anchore, Clair) = 0.63, Jaccard(Anchore, Vuls) =
	0.59, Jaccard(Vuls, Clair) = 0.80. Two Ubuntu 17.04 images weren't
	included in this comparison as they cannot be scanned by Vuls.}
\end{figure}

% In most cases,
% Anchore reports more vulnerabilities than Vuls and Clair resulting in 
% the Jaccard coefficient of 0.63 with Clair, 0.59 with Vuls,
% whereas Vuls and Clair share more common vulnerability detection resulting to the Jaccard coefficient of 0.86.
% Figure~\ref{fig:venn} represents the number of vulnerabilities reported by
% the scanners in Docker images and the overlap between them.
% This Figure also excludes two non-comparable images, which cannot be scanned with Vuls.
% It is interesting to note that out of all reported vulnerabilities, 4453 vulnerabilities are only
% reported by Anchore, 699 by Clair, and 536 by Vuls.}
% \newline\\
% As the number and type of vulnerabilities reported by the these scanners are significantly different from each other,
% we desire to investigate the reasons behind these differences.
% Following our investigation, we are able to explain some of the reasons behind
% the differences. 
% Such reasons were resulted from the way these scanners function or behave 
% in particular, the following actions by the scanners are indicative:
% \begin{itemize}
%    \item scanners ignoring vulnerabilities in Linux-kernel-header packages,
%    \item scanners referring to different vulnerability databases,
%    \item different frequency of updating their databases,
%    \item bugs in scanners. 
% \end{itemize}

% Below, we explain these reasons in further details and provide examples
% for clear illustrations.

% \textbf{Linux-kernel-headers:} Clair and Vuls ignore all vulnerabilities in the Linux-kernel-headers,
% whereas Anchore reports all of them.
% We find that 4443 out of 4453 vulnerabilities that are only reported by Anchore are present in Linux-kernel-headers.
% For example, CVE-2019-9506 present in Linux-libc-dev-4.4.0-31.50, reported by Anchore, is not
% detected by both Vuls and Clair.
% Linux-kernel-headers are the header files that only provide function names, their parameters, and their
% return types.
% The actual source code of those functions is present in the Linux kernel and is not inside the image.
% So, vulnerability detection in header files
% is irrelevant in this case.


% REGION (7)
% \textbf{Ubuntu vulnerabilities with status as “Not-affected”:} Vuls reports vulnerabilities that have status
% as "not-affected" in the Ubuntu database. It means that according to Ubuntu the image is not affected
% by the particular vulnerability. In other words, Vuls reports vulnerabilities that it should not have been reported.
% For example, when we scan an image having the Trusty release in it and Vuls reports CVE-2015-2305
% in the package Cups. However, upon checking the Ubuntu vulnerability database, we find that package Cups in Trusty is not
% affected by this vulnerability.

% HOW ABOUT VULS?
% \textbf{Ubuntu vulnerabilities with status as "Ignored (reached end-of-life)":}
% Some vulnerabilities are ignored in Ubuntu 14.04 LTS as it reached end-of-life. 
% However, in Ubuntu 14.04 ESM release, these vulnerabilities may or may not be present.
% Anchore checks the last status of these vulnerabilities in LTS whether the package was vulnerable
% prior to the end-of-life. Only if it was vulnerable, Anchore reports it, otherwise it does not.
% If this kind of vulnerability is present in ESM release then Clair reports it because it
% only refers to ESM release. On the other hand, if the vulnerability
% status in ESM release is "DNE" (Does Not Exist), then Clair does not report it. 
% For example, CVE-2017-9994 vulnerability in Libav package is ignored in Trusty LTS release due to its end-of-life.
% Consequently, it is not reported by Anchore. This vulnerability has status as "DNE" in Trusty ESM
% release so Clair also does not report it. As Ubuntu 14.04 LTS release is present inside the images,
% Clair's referring to Ubuntu 14.04 ESM release vulnerability database is incorrect.

% REGION (4)
% \textbf{Vulnerabilities not present in Ubuntu 14.04 ESM release:} These vulnerabilities are not present in Ubuntu 14.04 ESM so they 
% are not detected by Clair but they affect Ubuntu 14.04 LTS, and hence detected by Vuls and Anchore. The CVE-2017-13758 is an example
% of such vulnerabilities.

% CHECK THAT
% \textbf{Misses:} Clair and Vuls
% missed vulnerabilities that they should have detected. For example, the CVE-2016-9082 
% is present in package Cairo in Xenial release and it is not detected by Clair.
% Vuls also missed vulnerabilities such as CVE-2019-5094.

% REGION (3)
% \textbf{Wrong Detections:} 
% Some vulnerabilities are not linked to Ubuntu 14.04 LTS which is 
% present inside the image, however, Clair have reported such vulnerabilities.
% For instance, CVE-2019-1274 is an example of such vulnerabilities. 
% %is showing because these vulnerabilities are present in Ubuntu 14.04 ESM and 
% %Clair is only referring to Ubuntu 14.04 ESM release. 
% We reported this issue on two Google groups of Clair i.e. CoreOS Dev and
% clair-dev \url{(https://groups.google.com/forum/\#!topic/clair-dev/8IqCCfd-EEc).}

% \textbf{Database difference:} At the time we pulled Clair's database, through \texttt{arminc/clair-db:latest} Docker image, it was one week old.
%                               As a result some vulnerabilities were not updated in the database. For instance, CVE-2019-5094, which
% 		              was published on 2019-09-24, was not 
% 		              updated in Clair's database. Due to which it is missed by Clair.
% 			      Consequently, these discrepancies, even small, are sufficient to change the detection of
% 			      the scanners.

% REGION (6)
% \textbf{Epoch bug:} There is a bug in the Anchore scanner due to which some vulnerabilities are 
% 		not detected by Anchore. This bug gets triggered when a package’s version have epoch 
% 		(1:3.3.9-1ubuntu2.3) in it. For example, CVE-2018-1125 in Procps package is not
% 		detected by Anchore due to the presence of epoch in the Procps package version.
% 		We reported this bug to the Anchore development team through their Slack channel.

% \textbf{Ubuntu vulnerabilities with status "Ignored (out of standard support)" bug:} There is another bug in Anchore due to 
% 		which it is not able to detect some vulnerabilities. Due to this bug, Anchore does not detect 
% 		vulnerabilities that have a status as “ignored (out of standard support)” in Ubuntu 14.04 LTS. 
% 		However, if this vulnerability is present in Ubuntu 14.04 ESM release, it is detected by Clair. 
% 		For instance, CVE-2019-13565, which is present in Ubuntu 14.04 LTS, is an example of such vulnerabilities. 
% 		We reported this bug as well to the Anchore development team through their Slack
% 		channel.

% \textbf{Rejected CVEs:} There are some CVEs, such as CVE-2017-13753, that are rejected because they are duplicate copies of other CVEs. 
% 	So, Anchore and Clair does not report those rejected CVEs, but Vuls reports them.


% \tristan{integrate this part:

% \change{

% The major discrepancies are shown by Anchore and Clair in Ubuntu \tristan{why?}. Both refers to Ubuntu CVE Tracker database for Ubuntu images.
% The way in which both scanners interpret this database is different.
% %
% %There are different vulnerability databases, which have
% %different formats of data.
% %Out of which Open Vulnerability and Assessment Language (OVAL) is an international
% %open standard and a common
% %means of promoting publicly available security issues. 
% %It provides details about software vulnerabilities. It helps to determine
% %whether vulnerabilities exist in local machines or not. Apart from OVAL, operating system
% %vendors also keep track of their vulnerabilities separately and assign
% %a status to each vulnerability.
% %For example, Ubuntu maintains
% %Ubuntu CVE Tracker, Debian maintains Debian Security Bug Tracker, etc.
% %Each vulnerability is assigned a unique Common Vulnerabilities and Exposures (CVE) number
% %by the MITRE corporation, who maintains a list of all vulnerabilities of all distributions.
% %In this paper, we use vulnerability and CVE interchangeably.
% %Ubuntu CVE Tracker is used by Anchore and Clair in our experiments
% %and there is discrepancies in the way both scanners interpret results.
% %
% %BIDS apps and Boutiques images in our experiments have four Linux
% %distributions i.e. Ubuntu, CentOS, Debian, and Alpine. 
% %Out of these distributions, Ubuntu assigns a vulnerability status to
% %all vulnerabilities before keeping them in its vulnerability database.
% %Each vulnerability is assigned a unique Common Vulnerabilities and Exposures (CVE) number
% %by the MITRE corporation, who maintains a list of all vulnerabilities of all distributions.
% %Ubuntu keeps a record of Ubuntu-specific vulnerabilities in Ubuntu CVE Tracker using their CVE
% %numbers. In this paper, we use vulnerability and CVE interchangeably.
% %The vulnerability status in Ubuntu CVE Tracker is subjected to change
% %depending on the work progress to fix that vulnerability.
% %
% %Regarding Ubuntu, it has different
% %types of releases e.g. Regular, Long Term Support (LTS), and Extended Security Maintenance (ESM) release.
% %Ubuntu maintains its vulnerability data in Ubuntu CVE tracker, which is referred by Anchore and Clair.
% %There is another vulnerability database i.e. Open Vulnerability and Assessment Language (OVAL) repository,
% %which is maintained by MITRE corporation. Vuls refers to OVAL repository for getting Ubuntu vulnerability feed.
% %The vulnerability data format of both databases is different. OVAL keeps a separate xml file for each release,
% %which contains that release specific vulnerability information.
% Ubuntu CVE Tracker keeps a separate file for each vulnerability, which
% has an entry for all Ubuntu releases combined with the current status
% of that vulnerability in that release.
% In particular, Clair and Anchore show discrepancies for Ubuntu 14.04 which have
% two entries for each vulnerability, one for Ubuntu 14.04 Long Term Support (LTS)
% and second for Ubuntu 14.04 Extended Security Maintenance (ESM) release. As Ubuntu 14.04 LTS
% reached end-of-life on 2019-04-30, so Ubuntu ignores new vulnerabilities in LTS release and
% fix them only in ESM release. Consequently, a same vulnerability may have different vulnerability status in LTS and ESM releases,
% which is causing discrepancies. This is because Anchore refers to Ubuntu 14.04 LTS, whereas Clair refers to
% ESM.
% %Further, in Ubuntu a particular release can be of different types i.e. Regular, Long Term Support (LTS), and Extended Security Maintenance (ESM) release. 
% %Through our experiments, we find that Anchore and Clair report Ubuntu vulnerabilities 
% %differently depending on either the different Ubuntu release type or vulnerability status. 
% It is important to provide an overview of these Ubuntu vulnerability statuses, 
% since Ubuntu is heavily utilized in our experiments. 
% %Ubuntu's Regular release is supported only for 9 months, whereas LTS is supported for 5 years. Once LTS support
% %reaches End-Of-Life (EOL), Ubuntu decides whether to provide extended security support or not, which comes
% %in the form of ESM release.
% %ESM is a paid service that can be requested from Ubuntu to get security 
% %updates even after the EOL of a particular Ubuntu release. ESM and LTS releases of a particular Ubuntu version have different vulnerability
% %databases, because after EOL of a release, a vulnerability is fixed only in packages of ESM; however, it still
% %exists in LTS.
% %
% %As an example, Ubuntu has ended support for some releases, such as 14.04 LTS,
% %17.04, and 17.10, as they reached EOL. However, Ubuntu decided to provide optional extended support for Ubuntu 14.04.
% %To use this extended support, users have to purchase Ubuntu 14.04 ESM release.
% %The images in our experiments have Ubuntu 14.04 LTS, so scanners should refer
% %to LTS release only while collecting data from the vulnerability databases.
% %More interestingly, Clair refers to ESM release vulnerability database, however, this may lead to discrepancies in results.
% %Further, if an image is using a Linux release where EOL has been reached that image may
% %turns to be vulnerable as it is not possible to retrieve any updates.
% %
% In Ubuntu CVE Tracker, for a vulnerability all Ubuntu releases are listed with the package name
% in which that vulnerability exists and are entitled a vulnerability status
% (\url{https://git.launchpad.net/ubuntu-cve-tracker/plain/README}), which is
% encoded in the following form:
% \newline \\
% \noindent <release>\_<source-package>: <status> (<version/notes>)
% \newline\\
% The <version/notes> given with a vulnerability status provide more
% information about the vulnerability.
% For a given release, the vulnerability status can be any one of the following:

% \textbf{DNE (Does Not Exist):} The package does not exist in the
% 		release.

% \textbf{needs-triage:} The vulnerability of this package
% 		is not known, and hence evaluation is needed.

% \textbf{not-affected:} The package, while related to the
% 		vulnerability in some way, is not affected by the issue.
% 	%	The <notes> should
% 	%	provide detailed information, if needed.
% 		%For instance, if the given
% 		%status is "not-affected (1.14.6-1.1)" then this indicates that this specific
% 		%package version is not affected.

% \textbf{needed:} The package is vulnerable and needs to be fixed.

% \textbf{active:} The package is vulnerable, needs fixing, and is 
%                  actively being worked on.

% \textbf{ignored:} The package, while related to the
% 		vulnerability in some way, is being ignored for some reason. 
% 	%	The
% 	%	<notes> should provide that reason.
% 	%	For instance, if status looks like
% 	%	"ignored (reached end-of-life)", this means that the given Ubuntu release already reached
% 	%	end-of-life, so this CVE is ignored by Ubuntu. However, if there
% 	%	is extended security support for this release then this
% 	%	CVE will be handled in ESM release.

% \textbf{pending:} The package is vulnerable and
%                   has been fixed but an update has not been yet uploaded or
% 		  published.
% 		 % The <version> is given to indicate the particular 
% 		 % version where the fix has been done.

% \textbf{deferred:} The package is vulnerable, but 
%                    its fix has been deferred for some reason. 
% 		  % The <notes>
% 		  % should provide further details.
% 		  % If a date is mentioned, e.g.
% 		  % "deferred (2015-02-02)", the given date specifies the date when
% 		  % the CVE was deferred.

% \textbf{released:} The package was vulnerable, but
% 		an update has been already uploaded and published.
% 	%	e.g. "released (1.2.3)",
% 	%	<version> indicates the first version where the fix was applied.

% \textbf{released-esm:} The package was vulnerable and
% 		an update has been already uploaded and published. However,
% 		this update is published in the Ubuntu ESM release only and not in LTS.}
% 		%The fixed version of such packages is appended by either
% 		%+esm or \char`\~esm, indicating that this package version is available
% 		%only via ESM.

% 		% \newlne\\
% % Figure~\ref{example} illustrates how the information of a vulnerability is represented
% % in the Ubuntu CVE Tracker.

% % \begin{figure}[!ht]
% % 	\fbox{{\includegraphics[scale=2.5,width=\columnwidth]
% % 	{Figures/vulnExample2.png}}}
% %         \caption{\label{example} Representation of a vulnerability in Ubuntu CVE Tracker}
% % \end{figure}
% % \vspace*{-7mm}
% }


\section{Discussion}

There is a widespread issue with security vulnerabilities in container
images used for neuroimaging analyses, and it is likely to impact other
scientific disciplines as well. As shown in our results, it is common for container images
to hold hundreds of vulnerabilities, including several of critical
severity. Container images are impacted regardless of the type of analyses
that they support, and the main OS distributions Ubuntu, Debian and CentOS
are all affected.

Software updates remove about two-thirds of the vulnerabilities found and
should certainly be considered the primary solution to this problem.
However, in neuroimaging as in other disciplines, software updates are
generally discouraged because they can affect analysis results by
introducing numerical perturbations in the
computations~\cite{gronenschild2012effects,glatard2015reproducibility}. We
believe that this position is not viable from an IT security perspective,
and that it could endanger the entire Big Data processing infrastructure,
starting with the HPC centers. Instead, we advocate a more systematic
analysis of the numerical schemes involved in data analyses, which, coupled
with software testing, would make the analyses robust to software updates.
As a first step, the packages impacting the analyses could be specifically
identified and the others updated, which would largely remove
vulnerabilities.

Ultimately, software updates should even occur at runtime rather than when
the container image is built. Indeed, it is likely that container images
used for scientific data analyses be built only occasionally, perhaps every
few weeks when a release becomes available, which may not be compatible
with the frequency of required security updates. In fact, there is no
definite reason for the application software release cycle to be
synchronized with security updates, and security updates shouldn't be
dependent on application software developers. Instead, we think it would be
relevant for analytics engines to (1) systematically apply security updates
when containers start, and (2) run software tests provided by application
developers, including numerical tests, before running analyses. 

Implementing such a workflow, however, requires a long-term endeavour to
evaluate broadly the stability of data analysis pipelines, and to develop
the associated software tests. For the shorter term, we identified the
following recommendations for application developers to reduce the number
of security vulnerabilities in container images:

\begin{enumerate}[leftmargin=0pt,itemindent=*]
\item \emph{Introduce software dependencies cautiously}. Software
dependencies come with a potential security toll that is often neglected.
For instance, it can be tempting to add a complete toolbox to implement a
relatively minor operation in a data analysis pipeline, such as a data
format conversion, while the same functionality might be available in the
existing dependencies of the pipeline, albeit in a less convenient way.
\item \emph{Use lightweight base images such as Alpine Linux}. Base
images often come with packages that are useful in personal computers or
servers, but not in containers dedicated to a specific data analysis. In
addition, lightweight distributions define packages with a fine
granularity, allowing developers to avoid installing unnecessary
dependencies. For instance, the compressed size of the Alpine Linux base
image on DockerHub is only 2.7~MB while that of the Ubuntu base image is
27~MB.
\item \emph{Use OS releases with long-term support.} Security updates are
not provided for OS distributions that reached end of life. When a given
release of a data analysis pipeline is expected to be used over a long
period of time, typically several years as it is common in neurosciences,
the life cycle of the distribution release should be considered when
choosing a base container image. OS distributions have very different life
cycle durations, as long and short life cycles serve different purposes.
For instance, among RedHat-based distributions, Fedora release a new version
every 6 months and provide maintenance for about a year, while CentOS
release every 3-5 years and provide maintenance for 10 years. Similarly,
Ubuntu LTS (long-term support) distributions provide free security updates
for 5 years, and Debian stable releases are maintained for 3 years. 
% Commercial
% offers for extended maintenance periods are common. 
\item \emph{Install packages, not files}. Vulnerability scanners such as
Anchore, Clair or Vuls detect vulnerabilities from the list of packages
installed in a container image. Therefore, vulnerabilities contained in
software tools installed through direct file download rather than through
the package manager would go completely undetected. Domain-specific
distributions such as \href{http://neuro.debian.net}{NeuroDebian} or
\href{https://docs.fedoraproject.org/en-US/neurofedora/overview/}{NeuroFedora}
in neuroimaging are useful in this respect.
\item \emph{Minify container images.} The automated minification
process that we used in our study is unwieldy for a routine use, as it
requires capturing execution traces with ReproZip to reconstruct the graph
of package dependencies required for the analysis. In practice, it would be
more practical for software developers to identify and remove unnecessary
dependencies when they build containers, based on their knowledge of the
application.
\item \emph{Run image scanners during continuous integration.} Scanning
container images can be a cumbersome process that
could be asynchronously executed during continuous integration (CI),
through tools such as Travis CI or Circle CI. Including security scans in
CI also allows developers to identify vulnerabilities quickly,
before new software versions are released.
\end{enumerate}

Describing specific attacks against HPC systems that would exploit
vulnerabilities in container images is out of the scope of our study. We
believe that such attacks are likely to exist, although attacking HPC
systems through containers is challenging due to their relative isolation
from the host system. First, under the assumption that legitimate HPC
users can be trusted, attackers would have to be remote to the container,
either in the same network or on a remote network. Two main types of
attacks can be envisaged in these conditions: network-based attacks,
exploiting vulnerabilities in network clients installed in the container,
and data-based attacks, exploiting vulnerabilities through the processing
of malicious data injected through third-party systems.

Several types of escalation attacks could be envisaged once remote
attackers gain access to the container, in particular related to (1) using
the resources allocated to the container for malicious use, such as storing
data in the file system or using CPU cycles, resulting in denial of service
for the user running the container and possibly for other HPC users, and
(2) attacking a host network service, for instance a scheduler or a file
system daemon. Exploits in the host kernel to break out of the container
are always possible but unlikely assuming that the host system is
maintained by professional system administrators.

\section{Conclusion}

Most container images deployed on HPC clusters for scientific data analyses
contain hundreds of security vulnerabilities, many of which are critical.
In the short term, application software developers can address this issue
by: (1) minifying container images, by using lightweight OS distributions
and reducing software dependencies, and (2) applying regular security
updates, which requires using OS distributions with long-term support.
Longer term, data analysis pipelines would benefit from in-depth stability
analysis, to ensure that analytical results are not affected by security
updates.

This conclusion is not an alarming message urging HPC administrators  
to ban containers from their systems. User-controlled container images are
just one of many end-user artifacts that could serve as attack vectors, and
to our knowledge no attack has been described to exploit them. More
traditional types of attacks targeting user credentials or network
connections are likely to remain more common.

\bibliography{bibliography}
\end{document}

